\documentclass{llncs}   % list options between brackets

\usepackage{color}
\usepackage{graphicx}
%% The amssymb package provides various useful mathematical symbols
\usepackage{amssymb}
%% The amsthm package provides extended theorem environments
%\usepackage{amsthm}
\usepackage{amsmath}

\usepackage{listings}

\usepackage{hyperref}

\usepackage{systeme}

\def\shownotes{1}
\def\notesinmargins{0}

\ifnum\shownotes=1
\ifnum\notesinmargins=1
\newcommand{\authnote}[2]{\marginpar{\parbox{\marginparwidth}{\tiny %
  \textsf{#1 {\textcolor{blue}{notes: #2}}}}}%
  \textcolor{blue}{\textbf{\dag}}}
\else
\newcommand{\authnote}[2]{
  \textsf{#1 \textcolor{blue}{: #2}}}
\fi
\else
\newcommand{\authnote}[2]{}
\fi

\newcommand{\knote}[1]{{\authnote{\textcolor{green}{kushti notes}}{#1}}}

\usepackage[dvipsnames]{xcolor}
\usepackage[colorinlistoftodos,prependcaption,textsize=tiny]{todonotes}


% type user-defined commands here
\usepackage[T1]{fontenc}

\usepackage{flushend}


\newcommand{\bc}{Braid}

\newcommand{\ignore}[1]{} 

\begin{document}

\title{\bc{} - Bitcoin and Ergo Double Merge-Mined Sidechain with focus on Stablecoins, RWAs, Bitcoin DeFi}

\author{kushti \\ \href{mailto:kushti@protonmail.ch}{kushti@protonmail.ch}}


\maketitle

\begin{abstract}
\end{abstract}

\section{Introduction}

While disconnected from reality financial games, such as memecoins, still present, we clearly see solid emergent trends around utilizing 
blockchain-based assets for the needs of the real world:

* the most known blockchain-based asset, Bitcoin, is going into different corporate and national reserves. However, there Bitcoin becomes just another assets on the sheet for traditional financial schemes. There are attempts to build decentralized financial (DeFi) tooling on Bitcoin, and it was one of the biggest trends of 2024. There are two main directions in building DeFi on Bitcoin: somewhat L2 with EVM , or dedicated UTXO chain (Ergo, Nervos, Cardano) as execution environment for Bitcoin UTXOs progression, with Bitcoin UTXO set state delivered via trustless relays or trustless bridges, such as BitVM-based.

* there is big trend around stablecoins. Many private enterprises, following Tether's success, and nation states starting to issue own stablecoins, there are many efforts on making stablecoin payments seamless, reduce fees to almost zero. There are even buzzy at the moment 
announcements of stablecoin oriented blockchains, such as Plasma. 

While real-world adoption via different developments following the trends is highly positive, and moving forward a financial revolution we dreamed of, the state of developments is [...]


To tackle with the issues stated, we propose \bc{}, a blockchain which is oriented towards [...] 


\bc{} is a double merged-mined sidechain of both Bitcoin and Ergo, using Sigma, Ergo's contractual layer, along with modifications perfectly suitable for stablecoins and other real world assets. This would allow to have:

\begin{itemize}
  \item Proof-of-Work security from the biggest world's computing network (Bitcoin mining)
  \item Fees payable in any asset (you can even send gold-pegged tokens and pay in USD token, if there is option to swap USD for native token)
  \item Ready-made liqudity solutions since day one (!), such as bridges with Bitcoin, Ethereum, Binance Smartchain, Cardano
  \item Two-way trustless pegging with Ergo blockchain since day one
  \item Trustless (BitVM based, later BIP300/301 based maybe) bridge with Bitcoin
  \item RGB-like programmability on top of Bitcoin blockchain metadata
  \item Ready-made applications, such as AMM DEXes, orderbook DEXes, decentralized auctions, bonds, lending pools etc
  \item Regulatory sandboxing and compliance granularity, for example, to isolate jurisdiction-specific stablecoins, or shape real world assets usage with tailored compliance
  \item Compliance granularity may be combined with privacy
  \item Innovative algorithmic stablecoin designs, and insurance contracts, where algorithmic assets can insure non-delivery risks for tokenized real-world assets
  \item High performance: PoW secured blocks every few seconds
\end{itemize}



\section{Design}
\label{sec-design}




\section{Competition}
\label{sec-competition}


There are some attempts to build stablecoin and RWA focused blockchains, such as Tron and Plasma. However, usually they are just EVM chains with some features, like gasless transfers, implemented. Also, there are some attempts to build 

Here we propose a comprehensive set of solutions to Bitcoin DeFi, stablecoins, RWAs,
compliance granularity, precisely defined monetary circuits and so on.  

\bc{} is offering multi-layered set of solution for both Bitcoin DeFi and stablecoins and RWA. Unmatched support for money programmability allows stablecoin and RWA issuers to define rules of usage precisely. Trustless BTC pegging along with RGB like programmability allows for different kind of DeFi tooling for Bitcoin, including issuing trustless Bitcoin-backed derivatives. 


\section{Team}
\label{sec-team}

We have team of people participated in creation of such blockchains and blockchain projects as NXT (top3 CMC back then), Chainlink (top 20),
 Cardano (top 10), Waves (top 20 back then), Ergo (top 100 in 2021), and so on.


\section{Tokenomics and Liquidity}
\label{sec-tokenomics}

\knote{to be decided later}



\section{Technical Details}
\label{sec-techdetails}


\subsection{Global Transfer Policies}

In Braid, we want to augment Ergo contractual capabilities with Global Transfer Policies, set of contracts and limitation for a box, which may be propagated via transactions.

A Global Transfer Policy is set in a box, which needs to have a special NFT for identification. A policy box has locking script, like every box, which is allowing to change policies as well as locking script itself. A policy box should also have following registers:

\begin{itemize} 
 \item R4 - spending policy - any computation (getting the same context avaiable to a locking script, aside of mining pubkey and votes). Should return true or false. If true, the box may be spent, otherwise, not. 
 \item R5 - propagation policy - also computation - returns number of inputs which should have the same policy
\end{itemize}

Then we add another special register in Braid, and this register may contain multiple NFT ids. Boxes with such NFTs must be provided as read-only inputs of a transaction. An input can be spent if for all of its policies spending and propagation sub-policies satisfied.



Global Transfer Policies may have multiple use cases:

\begin{itemize}
\item a stablecoin issuer may use them to have black list or even white list. White list can be anonymized, by having stealth address like hiding in the white list
\item there could be very flexible policies  
\item they can be used to build different forms of Islamic Finance systems etc
\end{itemize}




\section{Implementation}
\label{sec-apps}



\newpage
\bibliography{sources}
\bibliographystyle{ieeetr} 

\end{document}
